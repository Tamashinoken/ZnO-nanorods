\documentclass[14pt]{article}
\usepackage{amssymb}
\usepackage[english,russian]{babel}
\usepackage{tikz}
\usepackage{href}
\usepackage{geometry}
\usepackage{amsmath}
\geometry{
	paper=a4paper,
	top=3.5cm,
	bottom=2.5cm,
	right=2cm,
	left=3cm
}
\begin{document}
	\begin{titlepage}
		\begin{center}
			\large
			O`ZBEKISTON RESPUBLIKASI\\ OLIY VA O`RTA-MAXSUS TA'LIM VAZIRLIGI
			MIRZO ULUG`BEK NOMIDAGI\\O`ZBEKISTON MILLIY UNIVERSITETI\\FIZIKA FAKULTETI\\FOTONIKA KAFEDRASI
			\vspace{0.25cm}
			\vfill
			\Large 			
			\Huge		
		%	\textsc{\textbf{Kurs ishi}}\\[5mm]
			{\Huge Rux oksidi asosidagi qattiq qotishmalarda fotolyuminessensiya hodisalari\\
			}
			\bigskip
		\end{center}
		\vfill
		\huge \ \  Magistrant: \_\_\_\_\_\_\ Saydusmonova Nihola
		
		\huge Ilmiy rahbar:\_\_\_\_\_\_\ Yuldashev Sh.U
		\vfill
		\begin{center}
			Toshkent, 2020
		\end{center}
	\end{titlepage}
	\newpage
	\renewcommand{\contentsname}{Mundarija}
	\tableofcontents
	\newpage
	\addcontentsline{toc}{section}{Kirish}
	\section*{Kirish}
	\hspace{1cm}	
	Hozirgi kunda ZnO ning yupqa plyonkalari, jumladan, ZnO:Al (AZO), ZnO:Ga(GZO), ZnO:In(IZO) lar o`zlarining ekologik tozaligi, toksik emasligi, arzonligi va boshqa optik hamda elektr xususiyatlari tufayli amaliyotda keng qo`llanilmoqda. Yuqorida sanab o`tilgan afzalliklari bilan ZnO quyosh batareyalari, kimyoviy sensorlar, lazer diodlari va opto-elektron ultrabinafsha nurli qurilmalarda ishlatiladi. To`g`ri ta'`qiqlangan zonasi  (3,37eV) va eksitonlarning bog`lanish energiyasi juda kattaligi(60 MeV) tufayli ZnO opto-elektron qurilmalar uchun asosiy tarkibiy qism hisoblanadi. Lekin ZnO ni amaliyotga keng tatbiq etilishi uning ta'qiqlangan zonasi kengligi tufayli shubha ostida qoladi. ZnO ning ta'qiqlangan zona kengligini modulyatsiyalash uchun qator tajribalar o`tkazilgan. Bular orasida Mg, Be va Cd elementlari bilan legirlash usuli ayniqsa, keng tarqalgan. ZnO ning hajmi bo`ylab kristall panjara tugunlaridagi Zn atomlarini MG, Be yoki Cd bilan almashtirish  orqali ZnO ning ta'qiqlangan zonasi muvaffaqiyatli tarzda modulyatsiyalandi. 
	
	Oxirgi vaqtlarda  Zn$_{1-x}$Mg$_{x}$ o`zining juda kichik ion radiusi Mg$^{2+}$ (0,57$\cdot 10^{-10}$m), Zn$^{2+}$ (0,60$\cdot 10^{-10}$ angstrem) hisobiga olimlar va ilmiy tadqiqotchilarda qiziqish uyg`otmoqda. Bundan tashqari uning ta'qiqlangan zonasi ham juda keng (7,80 eV). Zn$_{1-x}$Mg$_{x}$ ni  Mg bilan legirlashda uning ta`qiqlangan zonasi 3,37 eV dan 7,80 eV gacha o`zgaradi. ZnO/ZnMgO plyonkalarining ta'qiqlangan zonasi kerakli o`lchamgacha siljiganiga qaramay, ularning ultrabinafsha qurilmalar uchun o`tkazish koeffitsiyenti hamon bahsliligicha qolmoqda. Bundan tashqari Mg bilan legirlash orqali ZnO ning ta'qiqlangan zonasini o`zzgartirish uchun ko`p miqdorda Mg kiritilishi oqibatida plyonkalarning elektr xossalari pasayadi ($\rho\sim 1 \Omega$sm). 
	
	Shu sababli plyonkaning o`tkazuvchanligini saqlab qolgan holda, ta'qiqlangan zonasi kengligini o`zgartirish bo`yicha tadqiqolar olib borilmoqda. 
	
	Hozirgi vaqtda ZnMgO plyonkalarini impulsli lazer nuri bilan o`tqazish (PLD), molekulyar-nur epitaksiyasi (MBE), organik metall bug`larini o`tqazish (MOCVD), magnetron purkagichi (MS) va ultratovush purkagichli piroliz (USP) usullaridan foydalangan holda olinmoqda. Ular orasida USP metodi o`zining qulay va arzon ekanligi bilan ishlab chiqarish sohasiga tatbiq qilingan. 
	Ushbu tadqiqot ishida ham Zn$_{1-x}$Mg$_{x}$ molekulalarini plyonkalarga joylashtirish uchun USP -- ultratovush purkagichli piroliz usulidan foydalanilgan. Elektr xossalarini optik shaffoflik xossalari bilan birlashtirish uchun legirlanayotgan Mg miqdori nazorat qilib borilishi zarur. 
	
	 
	\section{ZnO va uning birikmalarining tuzilishi, xossalari}
	ZnO II-VI guruh yarimo`tkazgichlari toifasiga kiradi. Kristall panjarasi geksagonal vyursit ko`rinishida bo`ladi(\ref{fig:epm5}-rasm). Geksagonal panjara parametrlari: a=b=0,325 nm va c=0,520 nm. 
	
	ZnO ning tosh tuzi ko`rinishidagi kristall panjarasi(\ref{fig:epm5},b-rasm) kislorod va rux ionlarining orasidagi kuchli qutbli bog`lanish bilan tushuntiriladi. ZnO ning kovalent bog`i to`rtta teng sp$^{3}$ orbitallardan tashkil topgan. Ushbu orbitallar valent zonasini hosil qiladi. Anti orbitallar esa o`tkazuvchanlik sohasini hosil qiladi. 
	
	
\begin{figure*}[h]
	\centering
	\includegraphics[width=0.7\linewidth]{epm6}
	\caption{ZnO da nurlanishli o`tishlar}
	\label{fig:epm6}
\end{figure*}
	
	ZnO da nurlanishli o`tishlar \ref{fig:epm6}-rasmda tasvirlangan. 
	Rasmdan ko`rinib turibdiki, erkin elektronning o`tkazuvchanlik sohasining quyi sathidagi erkin kova bilan rekombinatsiyasi ta`qiqlangan zonaga mos keluvchi energiyali foton tug`ilishiga sabab bo`ladi. 
	
	Agar o`tish donor yoki akseptor holat orqali yuz  bersa, ushbu fotonning energiyasi kamayadi(\ref{epm6},b,c-rasm). Fotonlar energiyasining kamayishi yarimo`tkazgichda eksitonlarning vujudga kelishi bilan ham amalga oshishi mumkin. 
	
	Mos holda eksiton o`tishlar \ref{epm6},d,g-rasmlarda keltirilgan. Eksiton kvazizarra, elektron va kovakning vodorodsimon holati ko`rinishida o`zini tutadi. Bunda elektron kovakning atrofida aylanadi, deb qarash mumkin. 
	
	Agar eksiton fazoviy lokalizatsiyalanmagan bo`lsa, erkin eksiton deb qarash mumkin. Shunday qilib, rekombinatsiya natijasida nurlanadigan fotonlar, eksiton energiyasi E$_{x}$ miqdorida kamaygan ta`qiqlangan zona energiyasiga mos kelar ekan:
	\begin{equation}
	h\nu=E_{FX}=E_{g}-E_{x}
	\end{equation}
	bu yerda 
	\begin{equation}
	E_{x}=R_{y}^{*}\frac{1}{n^{2}}=13.6 eV\frac{\mu}{n^{2}\varepsilon^{2}}
	\end{equation}
	$\mu$ esa elektron-kovakning keltirilgan massasi. ZnO uchun $E_{x}$ ning qiymati xona haroratida 57 meV ni tashkil qiladi. 
	 \section{Eksperimentni o`tkazish}
	Indiy bilan legirlangan ZnMgO plyonkalari USP texnologiyalari asosida kvars plastinkalarga o`tqizildi. Bunda kvars 10 nm qalinlikda ZnO bilan qoplanadi. O`tqizish vaqtida temperaturani 420$^{\circ}$C qilib saqlab turiladi. Zn(Ac)$_{2}\cdot 2$H$_{2}$O, Mg(Ac)$_{2}\cdot 4$H$_{2}$O va In(NO$_{3}$)$_{3}$ eritmalari mos holda Zn, Mg va In ionlari uchun manba bo`lib xizmat qildi. 
	Optimallashtirilgan In/(Zn+Mg) nisbati 2,25\% ga teng bo`ldi. Yaxshi elektr xarakteristikalarini olish uchun legirlashni In bilan nazorat qilindi. Bundan tashqari Mg bilan legirlash uchun uning Zn$_{1-x}$Mg$_{x}$ plyonkalaridagi turli konsentratsiyalaridan foydalanildi: In (x=0; 0,5; 0,10; va 0,15). 
	
	
\begin{figure*}[h]
	\centering
	\includegraphics[width=0.7\linewidth]{epm5}
	\caption{ZnO ning kristall tuzilishi: geksagonal (a); kubik (b) va rux kristall panjarasi(c). Kichkina (qora) va katta (ko`k) doirachalar mos holda kristall panjara tugunidagi kation va anionlarni ifodalaydi }
	\label{fig:epm5}
\end{figure*}
	
	
	Dastlabki eritma aerozoli ultratovushli purkagichda hosil qilindi va 420$^{\circ}$C gacha qizdirilgan qoplamalarga yo`naltirildi. 
	
	Shundan so`ng hosil bo`lgan mahsulotlar Ar$_{2}$+H$_{2}$ (oqim nisbati 90:10) aralashmasli bilan to`ldirilgan pechda 400$^{\circ}$C temperaturada 20 minut davomida qizdirildi. Kristall strukturani rentgen nurlarining difraksiyasi orqali o`lchandi. Bunda XRD, D8-Focus qurilmasidan foydalanildi. Plyonkalarning morfologiyasi emission skanerlovchi elektron mikroskop (FESEM, Hitachi S-8010) qurilmasi orqali tekshirildi. Fluoressension spektrlar FluoroMax-4P spektrofluorimetri bilan olindi. Ushbu spektrometrda yorug`lik manbai sifatida nurlanish toqlin uzunligi 350 dan 1000 nm gacha bo`lgan UV-VIS-2450 spektrometridan foydalanildi. Elektr xossalari esa Van der Pau konfiguratsiyasida Xoll effektini o`lchash orqali aniqlandi. 
		
\section{Eksperiment natijalarini tahlil qilish}

\begin{figure*}[h]
	\centering
	\includegraphics[width=0.7\linewidth]{epm1}
	\caption{Zn$_{1-x}$Mg$_{x}$ plyonkalarining rentgenogrammalari. b - yupqa Zn$_{1-x}$Mg$_{x}$ plyonkalarda piklarning Mg konsentratsiyasiga bog`liq ravishda siljishi }
	\label{fig:epm1}
\end{figure*}

\ref{fig:epm1}-rasmda In bilan legirlangan Zn$_{1-x}$Mg$_{x}$ plyonkalarining 420 $^{\circ}$C temperaturada olingan rentgenogrammasi keltirilgan. Rasmdan ko`rinib turibdiki, Zn$_{1-x}$Mg$_{x}$ plyonkalarining kristall tuzilishi geksagonal strukturaga ega bo`lib, (0,0,2) orientatsiyaga ega. Barcha difraksion piklar (0,0,2) va (1,0,3) tekisliklar bilan biga geksagonal faza (JCPDS 36-1451) da juda yaxshi mos tushadi. Mg va In oksidlariga mos keladigan piklar aniqlanmadi. 

Aniqlanishicha, Mg va In atomlari kristall panjara tugunlaridagi Zn atomlari bilan o`rin almashishsa ham, kristall panjara chegaralariga ta'sir ko`rsatmaydi, ya'ni kristallning o`lchamlari o`zgarmaydi. \ref{fig:epm1},b-rasmda to`liq pik yarim maksimumining kengligi(FWHM) va (0,0,2) pikning Zn$_{1-x}$Mg$_{x}$ tarkibidagi Mg konsentratsiyasiga bog`liqligi  ko`rsatilgan. Grafikdan ko`rinib turibdiki, Mg ning konsentratsiyasi 0 dan 0,15 gacha ortganda FWHM pik (0,0,2) ning qiymati kamayadi. Biroq mos holda difraksiya burchagi Mg ning konsentratsiyasi katta bo`lgan sohada katta ekanligi ko`rinib turibdi. Ya'ni: $k=2d\sin\theta$. 
	
	Zn$_{1-x}$Mg$_{x}$ plyonkalarining rentgen tahlilil ko`rsatadiki, Zn kristall panjarasining tugunlari Mg va In elementlari bilan muvaffaqiyatli tarzda to`ldiriladi. Plyonkalarda Mg va In kiritilishi tufayli kristall panjara doimiyisining o`zgarishi tufayli cho`zilish deformatsiyasi paydo bo`ladi. 
	
	
	
\begin{figure*}[h]
	\centering
	\includegraphics[width=0.7\linewidth]{epm2}
	\caption{Mg ning turli konsentratsiyalarida Zn$_{1-x}$Mg$_{x}$ plyonkalarining FESEM da olingan tasviri: a) x=0,\  b) x=0,05, \ c) x=0,10,\  d) x=0,10, \ e)  namunalarning ko`ndalang kesimi morfologiyasi, x=0,15 da In va Mg elemantlarining taqsimoti (mos holda f va g)}
	\label{fig:epm2}
\end{figure*}


\ref{fig:epm2}-rasmda turli molyar nisbatdagi eritmalardan tayyorlangan namunalarning FESEM tasviri keltirilgan. Ko`rinib turibdiki, Mg ning eritma tarkibida bo`lishi hal qiluvchi rol o`ynaydi. 
\ref{fig:epm2},a-rasmdan ZnO:In plyonkalarida 100 nm o`lchamdagi zarralarning bir tekis taqsimlanganini ko`rishimiz mumkin. Biroq Mg ning massa ulushi ortib borgan sayin kristallarning 300 nm gacha kattalashishini ko`rishimiz mumkin(\ref{fig:epm2},b-rasm). 
Buni quyidagicha tushuntirish mumkin: Mg ning kiritilishi, USP paytida sirtdagi reaksiyalarni kuchaytiradi. Mg kiritilgandan keyin kristall o`lchamining kattalashishi FWHM natijalari bilan mos keladi. ZnO:In plyonkalarining ko`ndalang kesimi morfologiyasidan ko`rinadiki, plyonkaning qalinligi 300 nm ni tashkil etadi.(\ref{fig:epm2},d-rasm). Zn$_{1-x}$Mg$_{x}$:In plyonkalari kvars qoplama bilan bog`laniib ketgan. \ref{epm2},f va g-rasmlarda ko`rinib turibdiki, In va Mg elementlari sirtda bir jinsli taqsimlangan. Ularning massa ulushi mos holda 1,70 \% va 0,95\% ni tashkil etadi. 
	
\begin{figure*}[h]
	\centering
	\includegraphics[width=0.7\linewidth]{epm3}
	\caption{Zn$_{1-x}$Mg$_{x}$:In plyonkalarining fotolyuminessension spektri}
	\label{fig:epm3}
\end{figure*}
	\ref{fig:epm3}-rasmda Zn$_{1-x}$Mg$_{x}$:In plyonkalarining fluoressension spektri keltirilgan. Bunda intensivlik Mg ning konsentratsiyasiga bog`liq funksiya ko`rinishida tasvirlangan. Ma'lumki, ZnO ning fluoressension spektrida ikkita asosiy nurlanish chiziqlari mavjud bo`ladi. Bular NBE nurlanish va quyi sathlardan nurlanish. Biroq biz olgan fluoressension spektrda ultrabinafsha nurlanish sohasida joylashgan yaqqol ajralgan piklar mavjud. Bu piklar 378 nm dan 370 nm ga siljigan ko`k chiziqlar ekanligni e'tirof etishimiz mumkin. Zn$_{1-x}$Mg$_{x}$:In plyonkalaridagi defektlar tufayli quyi sathlar nurlanishi amalda umuman ko`rinmaydi. Mg va In atomlarining kiritilishi kristall panjara strukturasin buzmaydi. Mg kiritilgani uchun Zn$_{1-x}$Mg$_{x}$:In spektrining kengayishi kuzatiladi.
	
\begin{figure*}[h]
	\centering
	\includegraphics[width=0.7\linewidth]{epm4}
	\caption{Mg ning turli konsentratsiyalarida yupqa plyonkalarning o`tkazish spektri(a) va $\alpha$ qiymat va foto energiya orasidagi korrelyatsiya}
	\label{fig:epm4}
\end{figure*}
	\ref{fig:epm4},a-rasmda Mg ning turli miqdorlarida eritmalar o`tkazish spektrlari tasvirlangan. Barcha plyonkalarning o`rtacha o`tkazish koeffitsiyenti ko`zga ko`rinadigan nurlar sohasi (400-800 nm) uchun 90\% ni tashkil qiladi. Bu katta o`lchamli kristallar uchun kuda yaxshi o`tkazuvchanlik hisoblanadi.
	
\ref{fig:epm4},b-rasmda $\alpha$ kattalik qiymati va foton energiyasi(eV) orasidagi korrelyatsiya tasvirlangan. Bu yerda $\alpha$ -- yutilish koeffitsiyenti. Mg ning miqdori orttirilganda yutilish spektrining kichikroq to`lqin uzunliklari sohasiga siljiganini ko`rishimiz mumkin.  

\ref{fig:epm4}b-rasmga ilova qilingan rasmda optik ta'qiqlangan zona(E$_{g}$) tasvirlangan. Ushbu optik ta'qiqlangan zona ultrabinafsha nurlanish o`tkazilganda yutilishning boshlannish nuqtasi bilan aniqlanadi. Ko`rinib turibdiki, Mg ning qiymati 0,15 gacha orttirilganda E$_{g}$ ning qiymati ham ortib boradi. Grafikda x=0, 0,05, 010, 0,15 qiymatlar uchun mos holda ta`qiqlangan zonaning kengligi qiymatlari keltirilgan: 3,27, 3,29, 3,32, 3,37 eV. Bundan xulosa qilishimiz mumkinki, Mg kiritilishi plyonkaning ta`qiqlangan zona kengligini oshirar ekan. Ta`qiqlangan zona kengligining bunday ortishi hodisasi yarimo`tkazgichlardagi  Burshteyn-Moss siljishi bilan bog`liq. Bundan tashqari o`tkazish spektrining ko`k nurlanishga siljishi ultrabinafsha nurlanish spektri bilan yaxshi mos keladi. 
	
\newpage
\addcontentsline{toc}{section}{Xulosa}
\section*{Xulosa}
	\noindent	
	Zn$_{1-x}$Mg$_{x}$:In shaffof plyonkalari USP texnologiyalari asosida tayyorlandi va ularning ta`qiqlangan zona kengligi Mg bilan legirlash orqali modifikatsiyalandi. Mg ning miqdori ortib borishi bilan kristallning o`lchamlari kattalashib bordi.
	
	Mg ning massa ulushi 1,7\% bo`lganda eng yaxshi natijaga erishildi. Bu qiymat x=0,15 da Zn$_{1-x}$Mg$_{x}$ plyonkalarida kuzatildi. Barcha plyonkalar ko`zga ko`rinadigan nurlar sohasi (400-800 nm) uchun 90\%dan yuqori bo`lgan o`tkazuvchanlik koeffitsiyentiga ega bo`ldi.
	
	Mg kiritilishi tufayli ta`qiqlangan zona kengligi 3,27 eV dan 3,37 eV ga kengaydi. Fluoressensiya nurlanishi to`lqin uzunligi esa 378 nm dan 370 nm ga kamaydi. 
	
	Mg ning miqdori 0 dan 0,15 gacha o`zgartirilganda eritmalarda solishtirma qarshilik 6,70$\cdot 10^{-3}\Omega\cdot sm$ dan 2,14$\cdot 10^{-2}\Omega\cdot sm$ gacha o`zgardi. Harakatchanlik esa 24,7 sm$^{2}V^{-1}S^{-1}$ dan 6,46$sm^{2}V^{-1}S^{-1}$ gacha kamaydi. 
	
	Ushbu Zn$_{1-x}$Mg$_{x}$:In plyonkalar ultratovush diapazonida ishlaydigan fotoelektrik qurilmalarda qo`llash uchun juda yaxshi nomzod hisoblanadi.
	
	Ushbu kurs ishida olingan natijalar magistrlik dissertatsiya ishiga ham kiritiladi. Magistrlik dissertatsiya ishining bitta rejasi mana shu eksperimentga bag`ishlanadi.
	
	\newpage
\renewcommand{\refname}{Adabiyotlar ro`yxati}
\begin{thebibliography}{23}
		\bibitem{1}  W. Yan, J. Tan, W. Zhang, X.K. Meng, T. Lei, C.M. Li, X.W. Sun, Mater. Lett. 87 (2012) 
		\bibitem{2} Y.F. Wang, X.D. Zhang, Q. Huang, C.C. Wei, Y. Zhao, Sol. Energy Mater. Sol. Cells. 110 (2013) 94.
		\bibitem{3}  C. Guillén, J. Montero, J. Herrero, Appl. Surf. Sci. 264 (2013) 448.
		\bibitem{4}  S.C. Woo, J.G. Yoon, Solid State Commun. 152 (2012) 345.
		\bibitem{5}  S.J. Chang, T.J. Hsueh, I.C. Chen, B.R. Huang, Nanotechnology 19 (2008) 175502.
		\bibitem{6}  H.J. Zhou, J. Fallert, J. Sartor, R.J.B. Dietz, C. Klingshirn, H. Kalt, D. Weissenberger, D. Gerthsen, H.B. Zeng, W.P. Cai, Appl.
		Phys. Lett. 92 (2008) 132112.
		\bibitem{7}  W.L. Park, G.C. Yi, H.M. Jany, Appl. Phys. Lett. 79 (2001) 2022.
		\bibitem{8}  H.H. Huang, G.J. Fang, X.M. Mo, H. Long, H.N. Wang, S.Z. Li, Y. Li, Y.P. Zhang, C.X. Pan, D.L. Carroll, Appl. Phys. Lett. 101
		(2012) 223504.
		\bibitem{9}  C.Y. Peng, Y.A. Liu, W.L. Wang, J.S. Tian, L. Chang, Appl. Phys. Lett. 101 (2012) 151907.
		\bibitem{10}  X. Zhang, X.M. Li, T.L. Chen, C.Y. Zhang, W.D. Yu, Appl. Phys. Lett. 87 (2005) 092101.
		\bibitem{11}  A.D. Acharya, S. Moghe, R. Panda, S.B. Shrivastava, M. Gangrade, T. Shripathi, D.M. Phase, V. Ganesan, Thin Solid Films 525
		(2012) 49.
		\bibitem{12}  C. Yang, X.M. Li, Y.F. Gu, W.D. Yu, X.D. Gao, Y.W. Zhang, Appl. Phys. Lett. 93 (2008) 112114.
		\bibitem{13}  W.J. Kim, J.H. Leem, M.S. Han, Y.R. Ryu, T.S. Lee, J. Appl. Phys. 99 (2006) 096104.
		\bibitem{14}  R. Ghosh, D. Basak, J. Appl. Phys. 101 (2007) 113111.
		\bibitem{15}  I. Takeuchi, W. Yang, K.S. Chang, M.A. Aronova, T. Venkatesan, R.D. Vispute, L.A. Bendersky, J. Appl. Phys. 94 (2003) 7336.
		\bibitem{16}  C.R. Hall, L.V. Dao, K. Koike, S. Sasa, H.H. Tan, M. Inoue, M. Yano, C. Jagadish, J.A. Davis, Appl. Phys. Lett. 96 (2010) 193117.
		\bibitem{17}  H. Tampo, H. Shibata, K. Maejima, A. Yamada, K. Matsubara, P. Fons, S. Kashiwaya, S. Niki, Y. Chiba, T. Wakamatsu, H. Kanie,
		Appl. Phys. Lett. 93 (2008) 202104.
		\bibitem{18}  K. Fleischer, E. Arca, C. Smith, I.V. Shvets, Appl. Phys. Lett. 101 (2012) 121918.
		\bibitem{19}  J.G. Lu, Y.Z. Zhang, Z.Z. Ye, L.P. Zhu, L. Wang, B.H. Zhao, Q.L. Liang, Appl. Phys. Lett. 88 (2006) 222114.
		\bibitem{20}  H. Tampo, H. Shibata, K. Maejima, A. Yamada, K. Matsubara, P. Fons, S. Niki, T. Tainaka, Y. Chiba, H. Kanie, Appl. Phys. Lett.
		91 (2007) 261907.
		\bibitem{21}  J.D. Ye, S.L. Gu, W. Liu, S.M. Zhu, R. Zhang, Y. Shi, Y.D. Zheng, X.W. Sun, G.Q. Lo, D.L. Kwong, Appl. Phys. Lett. 90 (2007)
		174107.
		\bibitem{22}  C. Guillen, J. Herrero, Vacuum 82 (2008) 668.
		\bibitem{23}  T. Dedova, O. Volobujeva, J. Klauson, A. Mere, M. Krunks, Nanoscale Res. Lett. 2 (2007) 391.
		
\end{thebibliography}
	
	
	
	
	
	
	
	
	
	
	
	
	
	
	
	
	
	
	
	
	
	
	
	\end{document}
